\section*{Exercício 3}
\label{ex:3}
\addcontentsline{toc}{section}{Exercício 3}
    
Seja um sistema discreto LIT MIMO descrito da forma 

\begin{equation}
    \begin{cases} \mathbf{x}_{n+1} &= \mathbf{\Phi} \mathbf{x}_n \mathbf{+ \Gamma} u_n \\ \mathbf{y}_n &= \mathbf{C} \mathbf{x}_n \end{cases}
    \label{eq:LITdiscreto}
\end{equation}

tal que $\mathbf{\Phi} \in \mathbb{R}^{n}$, $\mathbf{\Gamma} \in \mathbb{R}^{n, m}$ e $\mathbf{C} \in \mathbb{R}^{n, m}$. O diagrama de blocos de um sistema discreto com controle proporcional-integrativo discretizado por diferenças finitas para frente como apresentado acima está em \cite[9, p. 172]{controle_digital_2018}. Sejam as equações de diferenças de um integrador e ação de controle dadas pela relações \eqref{eq:integrador} e \eqref{eq:acao_integrador}, o sistema extendido para um controlador com integrador é da forma apresentada em \eqref{eq:aug_system}.

    \begin{subequations}
        \begin{equation} \label{eq:integrador}
            \begin{split}
                \mathbf{v}_{n+1} & = \mathbf{v}_{n} + \mathbf{w}_{n} - \mathbf{y}_{n} \\
                                 & = \mathbf{v}_{n} + \mathbf{w}_{n} - \mathbf{C} \mathbf{x}_n
            \end{split}
        \end{equation} \\
        \begin{equation} \label{eq:acao_integrador}
            \mathbf{u}_n = - \mathbf{K} \mathbf{x}_n + \mathbf{K_i} \mathbf{v}_n
        \end{equation}
    \end{subequations}
    
    Para a ação de controle \eqref{eq:acao_integrador}, o sistema \eqref{eq:LITdiscreto} assume a forma

    \begin{equation}
        \begin{dcases}
            \begin{bmatrix} \mathbf{x}_{n+1} \\ \mathbf{v}_{n+1}\end{bmatrix} &= \begin{bmatrix} \mathbf{\Phi} & \mathbf{0}_{n, m} \\ -\mathbf{C} & \mathbf{I}_m \end{bmatrix} \begin{bmatrix} \mathbf{x}_{n} \\ \mathbf{v}_{n}\end{bmatrix} + \begin{bmatrix} \mathbf{\Gamma} \\ \mathbf{0}_{m}\end{bmatrix} \mathbf{u}_{n} + \begin{bmatrix} \mathbf{0}_{n, m} \\ \mathbf{I}_{m}\end{bmatrix} \mathbf{w}_{n} \\ 
            \mathbf{y}_{n} &= \begin{bmatrix} \mathbf{C} & \mathbf{0}_{n, m} \end{bmatrix} \vspace{5pt} \begin{bmatrix} \mathbf{x}_{n} \\ \mathbf{v}_{n}\end{bmatrix}
            \label{eq:aug_system}
        \end{dcases}
    \end{equation}

Deseja-se validar a proposição de que o sistema \eqref{eq:aug_system} é controlável se o sistema oiginal \eqref{eq:LITdiscreto}. Por meio do lema de \emph{Hautus} \cite[1, pg. 5]{mehrgroessen_2013} é possível validar a afirmação. O lema é dado informalmente a seguir

    \begin{definition}
        A matriz de controlabilidade de \emph{Hautus} para $\mathcal{S} - (\mathbf{\Phi}, \mathbf{\Gamma})$ é dada por $\mathcal{C}_H = \mathcal{C}_H(\mathbf{\Phi}, \mathbf{\Gamma}) = \left[ s \mathbf{I}_n - \mathbf{\Phi} \quad \mathbf{\Gamma} \right], \forall s \in \mathbb{C}$.
    \end{definition}

    \begin{lemma}[\citeauthor{Sontag90mathematicalcontrol}]
        \label{lemma:hautus}
        Seja um sistema $\mathcal{S} = (\mathbf{\Phi}, \mathbf{\Gamma}), $ $\mathbf{\Phi} \in \mathbb{R}^{n}$, $\mathbf{\Gamma} \in \mathbb{R}^{n, m}$, este é controlável se $\mathrm{rank} \, \mathcal{C}_H \, = n, \forall \, s \in \mathbb{C}$
    \end{lemma}

    \begin{proposition}
        Seja lema \ref{lemma:hautus}, se $\exists \lambda_i, \, \mathrm{rank} \, \mathcal{C}_H (\lambda_i) < n$, então este é um autovalor não-controlável de \eqref{eq:LITdiscreto} e consequentemente o sistema não é controlável. Por fim, se para todo $\lambda_i$, $\mathrm{rank} \,\, \mathcal{C}_H = n$, então o sistema é controlável. 
    \end{proposition}

    \begin{proof}
        \label{proof:hautus}
        A matriz $s \mathbf{I}_n - \mathbf{\Phi}$ não possui posto completo apenas para $s = \lambda_i, \, i = 1, \cdots, n$, para $\lambda_i$ as raízes do polinômio característico, nomeadamente autovalores de $\mathbf{\Phi}$. Desta forma, é suficiente validar o lema \ref{lemma:hautus} para os $n$ autovalores de $\mathbf{\Phi}$.
    \end{proof}

    \begin{proposition}
        Seja um sistema $\mathcal{S} = (\mathbf{\Phi}, \mathbf{\Gamma})$, se $\mathcal{S}$ é controlável, logo $ \mathcal{\hat S} = (\mathbf{\hat \Phi}, \mathbf{\hat \Gamma})$ também é controlável.
    \end{proposition}

    \begin{proof}
        Considere a matriz de controlabilidade de Hautus $\mathcal{C}_H(\mathbf{\hat \Phi}, \mathbf{\hat \Gamma})$ dada a seguir.
            
        \begin{equation}
            \label{eq:c_H_hat}
            \mathcal{\hat C}_H(\mathbf{\hat \Phi}, \mathbf{\hat \Gamma}) = \begin{bmatrix} s\mathbf{I}_{n + m} - \mathbf{\hat \Phi} & \mathbf{\hat \Gamma}\end{bmatrix} = \begin{bmatrix} \lambda_i \mathbf{I}_n - \mathbf{\Phi} + \mathbf{K} \mathbf{\Gamma} & \mathbf{\Gamma} \mathbf{K}_i & \mathbf{0}_{n, m} \\ \mathbf{C} & \lambda_i \mathbf{I}_{m} & \mathbf{I}_m \end{bmatrix}
        \end{equation}
    
    A matriz \eqref{eq:c_H_hat} é equivalente à decomposição dada em 
    
    \begin{equation}
        \mathcal{\hat C}_H(\mathbf{\hat \Phi}, \mathbf{\hat \Gamma}) = \underbrace{\begin{bmatrix} \lambda_i \mathbf{I}_n - \mathbf{\Phi} & \mathbf{0}_m & \mathbf{\Gamma} \\ \mathbf{C} & \mathbf{I}_{m} & \mathbf{0}_{m} \end{bmatrix}}_{\mathcal{C}_{H}(\mathbf{\Tilde{\Phi}}, \mathbf{\Tilde{\Gamma}})} \underbrace{\begin{bmatrix} \mathbf{I}_n & \mathbf{0}_m & \mathbf{0}_m \\ \mathbf{0}_{m, n} & \lambda_i \mathbf{I}_{m} & \mathbf{I}_{m} \\
        \mathbf{K} & -\mathbf{K}_i & \mathbf{0}_m
        \end{bmatrix}}_{\mathbf{\Xi}}
        \label{eq:decomopsition_hat_C}
    \end{equation}
    
    A demonstração utiliza a propriedade de posto $\mathrm{rank}(A, B) \leq \min(\mathrm{rank}(A), \mathrm{rank}(B))$. Para $\lambda_i \in \mathbb{C}$,  $\inf(\mathrm{rank} \, \Xi) = n + m$ devido à primeira e terceira colunas-bloco. Além disso, $\inf\mathrm{rank} \, \mathcal{C}_{H}(\Tilde{\Phi}, \Tilde{\Gamma}) = n + m$ é verdade se $\mathcal{S}$ é controlável para todo $\mathbf{\Phi}, \mathbf{\Gamma}$. Pela propriedade de posto já citada, é suficiente verificar que \eqref{eq:c_H_hat} possui posto $n+m$ i.e. se $\mathcal{S}$ é controlável, então $\mathcal{\hat S}$ também é controlável.
    
    \end{proof}
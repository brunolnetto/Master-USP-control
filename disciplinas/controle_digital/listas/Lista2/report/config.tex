%----------------------- Macros and Definitions --------------------------
% Cores definidas pelo usuario
\definecolor{mygreen}{RGB}{28,172,0}
\definecolor{mylilas}{RGB}{170,55,241}

% Imagens e eixos
\pgfplotsset{compat=1.10}
\usetikzlibrary{pgfplots.fillbetween, backgrounds}

% Configuração de orientacao do Tikz
\tikzset{
    reverseclip/.style={insert path={(current page.north east) --
                                     (current page.south east) --
                                     (current page.south west) --
                                     (current page.north west) --
                                     (current page.north east)}
    }
}

% Listar secoess
\setcounter{secnumdepth}{0}

% Configuracoes do matlab
\lstset{
    language=Matlab,%
    basicstyle=\ttfamily,
    frame=single,
    breaklines=true,%
    morekeywords={matlab2tikz},
    keywordstyle=\color{blue},%
    morekeywords=[2]{1}, keywordstyle=[2]{\color{black}},
    identifierstyle=\color{black},%
    stringstyle=\color{mylilas},
    commentstyle=\color{mygreen},%
    showstringspaces=false,
    numbers=left,%
    numberstyle={\tiny \color{black}},
    numbersep=9pt,
    emph=[1]{for,end,break},emphstyle=[1]\color{red}, %some words to emphasise
    %emph=[2]{word1,word2}, emphstyle=[2]{style},    
}

% Ambientes para formalidades matemáticas
\theoremstyle{plain}
\newtheorem{thm}{Teorema}[section] % reset theorem numbering for each chapter
\newtheorem{definition}[thm]{Definição}
\newtheorem{lemma}[thm]{Lema}
\newtheorem{proposition}[thm]{Proposição}
\newtheorem{corollary}[thm]{Corolário}

\setcounter{section}{1}
\setcounter{thm}{0}

\renewcommand\qedsymbol{$\blacksquare$}

% Equações 
\makeatletter
\newcommand*\@dblLabelI {}
\newcommand*\@dblLabelII {}
\newcommand*\@dblequationAux {}

\def\@dblequationAux #1,#2,%
    {\def\@dblLabelI{\label{#1}}\def\@dblLabelII{\label{#2}}}

\newcommand*{\doubleequation}[3][]{%
    \par\vskip\abovedisplayskip\noindent
    \if\relax\detokenize{#1}\relax
       \let\@dblLabelI\@empty
       \let\@dblLabelII\@empty
    \else % we assume here that the optional argument
          % has the required shape A,B
       \@dblequationAux #1,%
    \fi
    \makebox[0.5\linewidth-1.5em]{%
     \hspace{\stretch2}%
     \makebox[0pt]{$\displaystyle #2$}%
     \hspace{\stretch1}%
    }%
    \makebox[0.5\linewidth-1.5em]{%
     \hspace{\stretch1}%
     \makebox[0pt]{$\displaystyle #3$}%
     \hspace{\stretch2}%
    }%
    \makebox[3em][r]{(%
  \refstepcounter{equation}\theequation\@dblLabelI, 
  \refstepcounter{equation}\theequation\@dblLabelII)}%
  \par\vskip\belowdisplayskip
}
\makeatother

\makeatletter
\newcommand*{\currentname}{\@currentlabelname}
\makeatother

\renewcommand\lstlistlistingname{Scripts em Matlab}
\renewcommand{\lstlistingname}{Script}
\renewcommand{\theenumi}{\bf \Alph{enumi}}
\newcommand{\studentname}{Bruno H. L. N. Peixoto}
\newcommand{\exercisename}{Lista}
\newcommand{\subjectname}{PTC5611 Controle Digital de Sistemas Dinâmicos}
\newcommand{\uspid}{7206666}
\newcommand{\uspmail}{bruno.peixoto@usp.br}
\newcommand{\esnumber}{2}

\newcommand{\headerstyle}{\sffamily\bfseries\large}

\fancypagestyle{nofooter}{
  % Limpa o cabecalho e roda-pe
  \fancyhf{}
  \fancyhead[RO, LE]{\headerstyle \subjectname}
  \renewcommand{\headrulewidth}{1pt}
}

\setlength{\headheight}{14.0pt}
\pagestyle{fancy}

\fancyhead[LE, RO]{\headerstyle \subjectname}
\fancyhead[RE, LO]{\currentname}
\cfoot{\thepage \hspace{1pt} de \pageref{LastPage}}	
\renewcommand{\headrulewidth}{1pt}

\graphicspath{{./images/}}



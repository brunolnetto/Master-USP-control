\maketitle
    \thispagestyle{fancy}

    % https://tex.stackexchange.com/questions/99224/putting-text-at-the-side-of-a-tikzpicture
    O modelo pendular emerge na natureza em diversas aplicações, entre outros, guindastes e sistemas de orientação de foguetes. Além disso, ele é amplamente utilizado na literatura por consistir em um sistema não-linear e apresentar conceitos fundamentais da teoria de controle. 
    
    A planta proposta consiste em um sistema pendular sub-atuado bidimensional composto por um carro $\mathcal{B}_0$ de massa $m_0$ associado a um sistema de elos $\mathcal{B}_1$ e $\mathcal{B}_2$ com massas $m_1$ e $m_2$ respectivamente referentes ao sistema de coordenadas afixado aos centros de massa, solidários aos elos. Os elos $\mathcal{B}_1$ e $\mathcal{B}_2$ consistem respectivamente em uma barra retangular homogênea de comprimento $\ell_1$, largura $\mathrm{w}_1$ e espessura desprezível e uma barra cilíndrica homogênea de comprimento $\ell_2$ e raio $r_2$ (Tabela \ref{tab:dados_planta}).
    
            \begin{figure}[ht]
        \centering
        \begin{tikzpicture}[thick,>=latex,->]
    
            % Trajectory projection
            \draw[dashed] (4.2,0) arc (0:90:4.2);
    
            % Bar
            \draw[draw=black,fill=white] (3, 3) circle circle (.5cm);
            \draw[double distance=2mm, -] (3.15, 2.8) -- (0, 7);
            \draw[double distance=5mm, -] (0,0) -- (3.2, 3.2);
    
            % Car
            \draw[fill=white] (1.2,-1.0) -- (.5,0) arc(180:360:-0.5) -- (-1.2, -1.0) -- cycle;
            \draw[draw=black,fill=white] (0, 0) circle circle (.15cm);
            
            % Motor
            \draw[fill=white] (-3.1,-1.4) -- (-3.5, -0.5) arc(180:360:-0.5) -- (-4.9, -1.4) -- cycle;
            \draw[draw=black,fill=white] (-4, -0.5) circle circle (.5cm);
            \draw[draw=black,fill=white] (-4, -0.5) circle circle (.15cm);
                            
            % Wheels
            \draw (0.7, -1.2) circle (2mm);
            \draw (-0.7, -1.2) circle (2mm);
    
            % Base
            \draw[pattern=north east lines] (-5.7, -1.6) rectangle (4.7, -1.4);
                
            % Joints
            \draw[draw=black,fill=white] (0, 0) circle circle (.15cm);
            \draw[draw=black,fill=white] (3, 3) circle circle (.15cm);
            
            % Force
            \draw[->] (-1.5, 0) -- (-0.5, 0) node[anchor = south east]{$\vec F$};
            \draw[->] (-3, 0) -- (-4, 0) node[anchor = south west]{$-\vec F$};
            
            \draw [dashed, -] (0, 0) -- (1, 1);
            \draw [dashed, -] (0, 0) -- (0, -1);
            \draw [dashed, -] (3, 3) -- (4, 4);
            \draw [dashed, -] (3, 3) -- (0, 7);
            
            % Axis
            \draw[->] (2.6, 0) -- (3.6, 0) node[below]{\texttt{x}};
            \draw[->] (2.6, 0) -- (2.6, 1) node[above]{\texttt{y}};
            
            % Angle
            \draw[thick, ->] (0, -0.5) arc (-90:45:0.5); 
            \node at (1, 0) {\Large $\theta_1$};
            \draw[thick, ->] (3.5, 3.5) arc (45:135:0.64);
            \node at (3, 4) {\Large $\theta_2$};
            
            %\draw (current bounding box.south west) rectangle (current bounding box.north east);
        
        \end{tikzpicture}
        \caption{Desenho esquemático do modelo}
        \label{fig:pendulum}
    \end{figure}
    
    Considere que o sistema apresenta excitação externa respectiva ao sinal PWM $u$ aplicado no circuito de armadura do motor à esquerda (Figura \ref{fig:pendulum}). O motor não apresenta redução, possui resistência de armadura $R$, constantes de acoplamento eletromagnético $K_t$ e $K_e$. A indutância de armadura do circuito é considerada nula para fins práticos. O torque gerado aciona uma polia de diâmetro $D$ de inércia desprezível acoplada ao motor e consequentemente um cabo de aço estirado inextensível.
    
        \begin{table}[!ht]
        \begin{minipage}[b]{0.45\linewidth}\centering
                \begin{tabular}{|c|c|l|}
                    \hline Descrição & Unidade & Valor\\ \hline
                    $R_a$ & $\Omega$ & 0.6\\
                    $K_t$ & $\frac{Nm}{A}$ & 0.262\\
                    $K_e$ & $Vs$ & 0.108\\
                    $b_0$   & ${N s}/{\mathrm{m}}$ & 1.950\\
                    $b_1$   & ${N s}/{\mathrm{rad}}$ & 0.010\\
                    $b_2$   & ${N s}/{\mathrm{rad}}$ & 0.010\\
                    $m_0$ & g & 440\\
                    $m_1$ & g & 153\\
                    $m_2$ & g & 71\\
                    $I_1$ & ${\mathrm{Kg} \mathrm{m^2}}$ & 6.63 $10^{-4}$\\
                    $I_2$ & ${\mathrm{Kg} \mathrm{m^2}}$ & $\frac{m_2 \ell_2^2}{12}$\\
                    $r$ & mm & 22.6\\
                    $\ell_1$ & mm & 180\\ 
                    $\ell_2$ & mm  & 215\\
                    $\ell_1^g$ & mm & 164\\ 
                    $\ell_2^g$ & mm  & 107\\
                    \hline
                \end{tabular}
                \caption{Dados da planta}
                \label{tab:dados_planta}
        \end{minipage}
        \hspace{0.5cm}
        \begin{minipage}[b]{0.45\linewidth}
        \centering
                \begin{tabular}{|c|c|c|}
                    \hline Descrição & Unidade & Valor\\ \hline
                    $m$ & bits & 10 \\
                    $n$ & - & 2 \\
                    $f_s$ & Hz & 100 \\
                    $\mathbf{t}_s$ & s & $3.\mathbf{1}$\footnote{$\mathbf{1} $ consiste em uma matriz-coluna composta por 1's com dimensão adequada.} \\
                    $\mathbf{R}_w$ & - & 0.02 $\mathds{1}$\footnote{[$\mathds{1}$] corresponde à matriz identidade}  \\ 
                    $\mathbf{R}_v$ & - & 0.02 $\mathds{1}$ \, \\
                    $\mathbf{\sigma}_u$ & $N$ & $500$ \\
                    $t_{\delta}$ & s & $10$ \\
                    \hline
                \end{tabular}
                \caption{Dados do microcontrolador e sensores}
                \label{tab:dados_controle}
        \end{minipage}
    \end{table}
        
    Considere uma representação possível de estados $\mathbf{x} = [x, \theta_1, \theta_2, \dot x, \dot \theta_1, \dot \theta_2]^T$ e uma das posições de repouso (instável) definida por $\mathbf{x}_{0} = [0, \pi, 0, 0, 0, 0]^T$. O controle linear pretende permanecer em uma região de convergência $\chi$ suficientemente próxima a $\mathbf{x}_{0}$ a fim de garantia linear do modelo \cite{Adamy2009} tal que o sistema permaneça em repouso a proposta consiste no projeto de um controlador LQR com frequência de amostragem $f_s$. O sistema controlado deve apresentar tempo de assentamento menor que $\mathbf{t}_s^{1\%}$ para cada um dos estados do sistema e desvio-padrão para a força aplicada igual a $\mathbf{\sigma}_u$. 
    
    Os sensores disponíveis são encoderes rotativos para o motor de acionamento i.e. leitura de $\mathbf{x}$, e um para leitura de $\theta_2$ acoplado à junta. As leituras dos sensores são decorrelacionadas e apresentam densidade espectral de potência $\mathbf{R}_v$. A posição angulares $\theta_1$ e $\theta_2$ apresentam ruído branco e a densidade espectral de potência iguais a $\mathbf{R}_w$ \cite{controle_digital_2018}. O microcontrolador escolhido apresenta atraso de cálculo de $n$ períodos de amostragem, $n \in \mathbb{N}$ e resolução de quantização de $m$ bits (Tabela \ref{tab:dados_controle}).
    
    %Como atividade extra, propõe-se considerar que o sistema seja submetido a um impulso $I = k \delta(t)$ \cite{Beer:2003:VME:1207649}, com $\delta(t)$ a função delta de Dirac, no instante $t_{\delta}$ na extremidade da segunda barra após sua estabilização a fim de determinar empiricamente $k$ tal que que o sistema não saia da região de estabilização.  
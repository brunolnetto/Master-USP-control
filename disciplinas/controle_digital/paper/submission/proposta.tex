\documentclass[a4paper, twoside, fleqn]{article}

% Pacotes utilizados
\usepackage[backend=biber, style=alphabetic]{biblatex}
\usepackage{amssymb, amsfonts, amsmath}
\usepackage[brazilian]{babel}
\usepackage[utf8]{inputenc}
\usepackage[brazilian]{babel}
\usepackage[utf8]{inputenc}
\usepackage{indentfirst}
\usepackage{lscape}
\usepackage{csquotes}
\usepackage{graphicx}
\usepackage{fancyhdr}
\usepackage{dsfont}
\usepackage{tikz}

% Arquivo da bibliografia 
\addbibresource{bibliography.bib}
\usetikzlibrary{patterns}

% Cabecalho
\newcommand{\studentname}{Bruno Peixoto}
\newcommand{\subjectname}{\hspace{10pt} PTC5611 Controle digital de sistemas dinâmicos}
\newcommand{\uspid}{7206666}
\newcommand{\uspmail}{bruno.peixoto@usp.br}
\newcommand{\esnumber}{1}
\newcommand{\headerstyle}{\sffamily \bfseries \small}
\renewcommand{\headrulewidth}{1pt}
\renewcommand{\headrulewidth}{1pt}

% Cabecalho
\fancyhead[RO, LE]{\headerstyle \subjectname}
\fancyhead[RE, LO]{}

\pagestyle{fancy}

% Título
\title{Proposta de projeto}
\author{\studentname \qquad \uspid \qquad \uspmail}

\begin{document}
    
    \maketitle
    \thispagestyle{fancy}

    % https://tex.stackexchange.com/questions/99224/putting-text-at-the-side-of-a-tikzpicture
    O modelo pendular emerge na natureza em diversas aplicações, entre outros, guindastes e sistemas de orientação de foguetes. Além disso, ele é amplamente utilizado na literatura por consistir em um sistema não-linear e apresentar conceitos fundamentais da teoria de controle. 
    
    A planta proposta consiste em um sistema pendular sub-atuado bidimensional composto por um carro $\mathcal{B}_0$ de massa $m_0$ associado a um sistema de elos $\mathcal{B}_1$ e $\mathcal{B}_2$ com massas $m_1$ e $m_2$ respectivamente referentes ao sistema de coordenadas afixado aos centros de massa, solidários aos elos. Os elos $\mathcal{B}_1$ e $\mathcal{B}_2$ consistem respectivamente em uma barra retangular homogênea de comprimento $\ell_1$, largura $\mathrm{w}_1$ e espessura desprezível e uma barra cilíndrica homogênea de comprimento $\ell_2$ e raio $r_2$ (Tabela \ref{tab:dados_planta}).
    
        \begin{figure}[ht]
        \centering
        \begin{tikzpicture}[thick,>=latex,->]
    
            % Trajectory projection
            \draw[dashed] (4.2,0) arc (0:90:4.2);
    
            % Bar
            \draw[draw=black,fill=white] (3, 3) circle circle (.5cm);
            \draw[double distance=2mm, -] (3.15, 2.8) -- (0, 7);
            \draw[double distance=5mm, -] (0,0) -- (3.2, 3.2);
    
            % Car
            \draw[fill=white] (1.2,-1.0) -- (.5,0) arc(180:360:-0.5) -- (-1.2, -1.0) -- cycle;
            \draw[draw=black,fill=white] (0, 0) circle circle (.15cm);
    
            % Wheels
            \draw (0.7, -1.2) circle (2mm);
            \draw (-0.7, -1.2) circle (2mm);
    
            % Base
            \draw[pattern=north east lines] (-1.7, -1.6) rectangle (1.7, -1.4);
    
            % Joints
            \draw[draw=black,fill=white] (0, 0) circle circle (.15cm);
            \draw[draw=black,fill=white] (3, 3) circle circle (.15cm);
            
            % Force
            \draw[->] (-1.5, 0) -- (-0.5, 0) node[anchor=south east]{$\vec F$};
            
            \draw [dashed, -] (0, 0) -- (1, 1);
            \draw [dashed, -] (0, 0) -- (0, -1);
    
            \draw [dashed, -] (3, 3) -- (4, 4);
            \draw [dashed, -] (3, 3) -- (0, 7);
            
            % Axis
            \draw[->] (2.6, 0) -- (3.6, 0) node[below]{\texttt{x}};
            \draw[->] (2.6, 0) -- (2.6, 1) node[above]{\texttt{y}};
            
            % Angle
            \draw[thick, ->] (0, -0.5) arc (-90:45:0.5); 
            \node at (1, 0) {\Large $\theta_1$};
            \draw[thick, ->] (3.5, 3.5) arc (45:135:0.64);
            \node at (3, 4) {\Large $\theta_2$};
            
            \draw (current bounding box.south west) rectangle (current bounding box.north east);
        
        \end{tikzpicture}
        \caption{Desenho esquemático do modelo}
        \label{fig:pendulum}
    \end{figure}
    
    Considere que o sistema apresenta excitação externa respectiva a uma força $\mathbf{F} = [F, 0, 0]^T$ aplicada ao corpo $\mathcal{B}_0$ (Figura \ref{fig:pendulum}) por meio de uma polia de diãmetro $D$ de inércia desprezível ligada a um motor distante de correia estirada inextensível e um momento gerado por um motor balanceado $\mathbf{M} = [0, 0, \tau]^T$ aplicado ao primeiro elo na junta que conecta $\mathcal{B}_1$ e $\mathcal{B}_2$. A correia de acionamento envolve a polia em um ângulo $\beta$ e possui coeficiente de atrito seco $\mu$. Para este projeto, não será realizado a modelagem do acionamento: a propoulsão é realizada pelo cabo e o sensoreamento é fornecido por $\mathrm{x}$. O vínculo do carro permite que o deslocamento seja apenas horizontal.
    
    \begin{table}[!ht]
        \begin{minipage}[b]{0.45\linewidth}\centering
                \begin{tabular}{|c|c|l|}
                    \hline Descrição & Unidade & Valor\\ \hline
                    $m_0$ & g & 440\\
                    $m_1$ & g & 153\\
                    $m_e$ & g & 8\\
                    $m_2$ & g & 7\\
                    $b_0$   & ${N s}/{\mathrm{m}}$ & 1.950\\
                    $b_1$   & ${N s}/{\mathrm{rad}}$ & 0.010\\
                    $b_2$   & ${N s}/{\mathrm{rad}}$ & 0.010\\
                    $\ell_1$ & mm & 180\\ 
                    $w_1$ & mm & 40\\
                    $e_1$ & mm & 40\\
                    $\ell_2$ & mm  & 215\\
                    $r_2$ & mm & 8\\ 
                    \hline
                \end{tabular}
                \caption{Dados da planta}
                \label{tab:dados_planta}
        \end{minipage}
        \hspace{0.5cm}
        \begin{minipage}[b]{0.45\linewidth}
        \centering
                \begin{tabular}{|c|c|c|}
                    \hline Descrição & Unidade & Valor\\ \hline
                    $m$ & bits & 10 \\
                    $n$ & - & 5 \\
                    $f_s$ & Hz & 100 \\
                    $\mathbf{t}_s$ & s & $3.\mathbf{1}$\footnote{$\mathbf{1} $ consiste em uma matriz-coluna composta por 1's com dimensão adequada.} \\
                    $\mathbf{R}_w$ & - & 0.02 $\mathds{1}$\footnote{[$\mathds{1}$] corresponde à matriz identidade}  \\ 
                    $\mathbf{R}_v$ & - & 0.02 $\mathds{1}$ \, \\
                    $\mathbf{\sigma}_u$ & $N$ & $500$ \\
                    $t_{\delta}$ & s & $10$ \\
                    \hline
                \end{tabular}
                \caption{Dados do microcontrolador e sensores}
                \label{tab:dados_controle}
        \end{minipage}
    \end{table}
    
    Sejam os estados do sistema $\mathbf{x} = [x, \theta_1, \theta_2, \dot x, \dot \theta_1, \dot \theta_2]^T$ e uma das posições de repouso (instável) definida por $\mathbf{x}_{0} = [0, \pi, 0, 0, 0, 0]^T$ tal que o sistema permanece em repouso e região de convergência $\chi$ suficientemente próxima a $\mathbf{x}_{0}$ a fim de garantia linear do modelo \cite{Adamy2009}, a proposta consiste no projeto de um controlador LQG com frequência de amostragem $f_s$. O sistema controlado deve apresentar tempo de assentamento menor que $\mathbf{t}_s^{1\%}$ para cada um dos estados do sistema e desvio-padrão para a força aplicada igual a $\mathbf{\sigma}_u$. 
    
    Os sensores disponíveis são um encoder rotativo para o motor de acionamento i.e. leitura de $\mathbf{x}$, e um para leitura de $\theta_1$ acoplado à junta, considerado uma massa pontual $m_e$. As leituras dos sensores são decorrelacionadas e apresentam densidade espectral de potência $\mathbf{R}_v$. A posição angulares $\theta_1$ e $\theta_2$ apresentam ruído branco e a densidade espectral de potência iguais a $\mathbf{R}_w$ \cite{controle_digital_2018}. O microcontrolador escolhido apresenta, em média, atraso de cálculo de $n$ períodos de amostragem, $n \in \mathbb{N}$, a depender do \emph{hardware}, e resolução de quantização de $m$ bits (Tabela \ref{tab:dados_controle}).
    
    Como atividade extra, propõe-se considerar que o sistema seja submetido a um impulso $I = k \delta(t)$ \cite{Beer:2003:VME:1207649}, com $\delta(t)$ a função delta de Dirac, no instante $t_{\delta}$ na extremidade da segunda barra após sua estabilização a fim de determinar empiricamente $k$ tal que que o sistema não saia da região de estabilização.  

\newpage

\section*{Apêndice}

Um sistema dinâmico satisfaz a equação de Lagrange dada por 

\begin{equation}
    \frac{\mathrm{d}}{\mathrm{dt}} \left( \frac{\partial L}{\partial \mathbf{\dot q}} \right) - \frac{\partial L}{\partial \mathbf{q}} + \frac{\partial F}{\partial \mathbf{\dot q}} = \mathbf{F_q}
\end{equation}

    tal que $\mathbf{q}$ e $\mathbf{\dot q}$ são as coordenadas generalizadas do sistema e suas respectivas derivadas, $L$ o lagrangiano dado por $L = T - U$, $T$ sua a energia cinética e $U$ a energia potencial do sistema, $F$ a energia dissipativa de Rayleigh e $\mathbf{F_q}$ as forças generalizadas do sistema. As forças generalizadas $\mathbf{F_q}$ advém do princípio dos trabalhos virtuais \cite{Beer:2003:VME:1207649} e satisfazem a relação $\mathbf{F_q} \delta{\mathbf{q}} + \mathbf{F}\delta \mathbf{x} = \delta \mathbf{W} \stackrel{\mathrm{def}}{=} 0$, para $\mathbf{x}$ os deslocamentos infinitesimais das posições de aplicação das excitações externas tal que $\mathbf{x} = \mathbf{x}(\mathbf{q})$.
    
    Umas das formas de representação de um sistema mecânico segue por $\mathbf{M}(\mathbf{q}) \mathbf{\ddot q} + \mathbf{\nu}(\mathbf{q}, \mathbf{\dot q}) + \mathbf{f}(\mathbf{q}, \mathbf{\dot q}) + \mathbf{g}(\mathbf{q}) = \mathbf{U}(\mathbf{q}) \mathbf{u}$. As matrizes $\mathbf{M}(\mathbf{q})$, $\mathbf{\nu(\mathbf{q}, \mathbf{\dot q})}$, $\mathbf{f}(\mathbf{\dot q})$ e $\mathbf{g}(\mathbf{q})$ são a matriz de massa acoplada, termos giroscópicos, atrito e de campo do sistema respectivamente. A matriz $\mathbf{U}(\mathbf{q})$ satisfaz a relação $\mathbf{F_q} = \mathbf{U}(\mathbf{q})\mathbf{u}$, tal que $\mathbf{u}$ são as excitações externas ao sistema.

    A representação em espaço de estados $\begin{cases} \mathbf{\dot x} = \mathbf{f}(\mathbf{x}, \mathbf{u}) \\ \mathbf{y} = \mathbf{g}(\mathbf{x}, \mathbf{u}) \end{cases}$ é dada por \eqref{eq:states}. Os estados do sistema são $\mathbf{x} = \begin{bmatrix} \mathbf{q} \\ \mathbf{\dot q} \end{bmatrix}$ e os sensores são representados genericamente pela relação $\mathbf{y} = \mathbf{g}(\mathbf{x}, \mathbf{u})$.
    
    \begin{equation}
    \label{eq:states}
        \begin{aligned}
            \frac{d}{dt} \begin{bmatrix} \mathbf{q} \\ \mathbf{\dot q} \end{bmatrix} &= 
            \underbrace{\begin{bmatrix}
                \mathbf{\dot q} \\
                - \mathbf{M}^{-1}(\mathbf{q}, \mathbf{\dot q})(\mathbf{\nu}(\mathbf{q}, \mathbf{\dot q}) + \mathbf{f}(\mathbf{q}, \mathbf{\dot q}) + \mathbf{g}(\mathbf{q}) )
            \end{bmatrix}}_{\mathbf{a}(\mathbf{x})} + 
            \underbrace{\begin{bmatrix} 
                0 \\ \mathbf{M}^{-1}(\mathbf{q}, \mathbf{\dot q}) \mathbf{U} 
            \end{bmatrix}}_{\mathbf{b}(\mathbf{x})} \mathbf{u}
        \end{aligned}
    \end{equation}
    
    Para um ponto de operação $(\mathbf{\dot x}_0, \mathbf{x_0})$, existe $\mathbf{u_0}$ tal que $\mathbf{\dot x_0} = \mathbf{a}(\mathbf{x}_0) + \mathbf{b}(\mathbf{x}_0) \mathbf{u_0}$, o sistema linearizado é dado por 
    
    \begin{subequations}
        \begin{equation}
            \begin{aligned}
                \Delta \mathbf{\dot x} &= \underbrace{(\frac{\partial \mathbf{a}}{\partial \mathbf{x}}(\mathbf{x_0}) + \frac{\partial \mathbf{b}}{\partial \mathbf{x}}(\mathbf{x_0}) \mathbf{u}_0)}_{\mathbf{A}(\mathbf{x_0}, \mathbf{u_0})} \Delta \mathbf{x} + \underbrace{ \mathbf{b}(\mathbf{x_0}) }_{\mathbf{B}(\mathbf{x_0}, \mathbf{u_0})} \Delta \mathbf{u} \\
                & = \mathbf{A}(\mathbf{x_0}, \mathbf{u_0}) \Delta \mathbf{x} + \mathbf{B}(\mathbf{x_0}, \mathbf{u_0}) \Delta \mathbf{u}
            \end{aligned}
        \end{equation}
        \begin{equation}
            \begin{aligned}
                \Delta \mathbf{y} &= \underbrace{\frac{\partial \mathbf{g}}{\partial \mathbf{x}}(\mathbf{x_0}, \mathbf{u_0})}_{\mathbf{C}(\mathbf{x_0}, \mathbf{u_0})} \Delta \mathbf{x} + \underbrace{\frac{\partial \mathbf{g}}{\partial \mathbf{u}}(\mathbf{x_0}, \mathbf{u_0})}_{\mathbf{D}(\mathbf{x_0}, \mathbf{u_0})} \Delta \mathbf{u} \\
                & = \mathbf{C}(\mathbf{x_0}, \mathbf{u_0}) \Delta \mathbf{x} + \mathbf{D}(\mathbf{x_0}, \mathbf{u_0}) \Delta \mathbf{u}
            \end{aligned}
        \end{equation}    
    \end{subequations}
    
    Sendo $\Delta \mathbf{\dot x} = \mathbf{\dot x} - \mathbf{\dot x}_0$, $\Delta \mathbf{x} = \mathbf{x} - \mathbf{x_0}$, $\Delta \mathbf{y} = \mathbf{y} - \mathbf{y_0}$ e $\Delta \mathbf{u} = \mathbf{u} - \mathbf{u_0}$. 
    
    Para o sistema representado na figura \ref{fig:pendulum}, as coordenadas generalizadas são $q = [x, \, \theta_1, \, \theta_2]^T$ e as matrizes características seguem abaixo. A matriz de massa do sistema é dada por \eqref{eq:mass_matrix}

    \begin{equation}
        M(\mathbf{q}) = 
        \begin{bmatrix}
            M_{11} & M_{11} & M_{13} \\
            M_{21} & M_{22} & M_{23}\\
            M_{31} & M_{23} & M_{33}
        \end{bmatrix}
        \label{eq:mass_matrix}
    \end{equation}
    
    tal que
    
    \begin{subequations}
         \begin{align}
            M_{11} &= \mathrm{m_0} + \mathrm{m_1} + \mathrm{m_2} \\
            M_{33} &= \mathrm{m_2}\, {\ell_{2}^g}^2\, + \mathrm{I_2} \\
            M_{22} &= \mathrm{m_1}\, {\ell_{1}^g}^2 + \mathrm{m_2}\, ({\mathrm{\ell_1}}^2 + {\ell_{2}^g}^2) + \mathrm{I_1} + \mathrm{I_2} + 2\, \mathrm{m_2}\, \mathrm{\ell_1}\, \ell_{2}^g\, \cos\!\left(\theta_2\right) \\
            M_{12} = M_{21} &= \mathrm{m_2}\, \ell_{2}^g\, \cos\!\left(\theta_1 + \theta_2\right) + (\mathrm{m_2}\,\mathrm{\ell_1} + \mathrm{m_1}\,\ell_{1}^g\,) \cos\!\left(\theta_1\right) \\
            M_{13} = M_{31} &= \mathrm{m_2}\, \ell_{2}^g\, \cos\!\left(\theta_1 + \theta_2\right) \\
            M_{23} = M_{32} &= \mathrm{m_2}\, {\ell_{2}^g}^2 + \mathrm{I_2} + \mathrm{m_2}\,\mathrm{\ell_1}\, \ell_{2}^g \, \cos\!\left(\theta_2\right)
         \end{align}
         \phantom{\hspace{6cm}}
    \end{subequations}
    
    A matriz de efeitos giroscópicos é dada por $\mathbf{\nu}(\mathbf{\dot q}, \mathbf{q}) = \begin{bmatrix}  \nu_1 \\ \nu_2 \\\nu_3 \end{bmatrix}$ tal que os componentes $\nu_i$ são
    
    \begin{equation}
        \begin{aligned}
             \nu_1 = & -\mathrm{m_2} \, \ell_{2}^g\, \sin\!\left(\theta_1 + \theta_2\right)\, (\mathrm{\dot \theta_1} + \mathrm{\dot \theta_2})^2 - (\mathrm{m_2} \mathrm{\ell_1} + \mathrm{m_1}\,\ell_{1}^g)\sin\!\left(\theta_1\right)\, {\mathrm{\dot \theta_1}}^2\\ 
             \nu_2 = & - \mathrm{m_2}\, \mathrm{\dot \theta_2}\, \left(2\, \mathrm{\dot \theta_1} + \mathrm{\dot \theta_2}\right) \, \ell_{2}^g\, \mathrm{\ell_1}\, \sin\!\left(\theta_2\right) \\
             % todo Checar sinal do termo abaixo
             \nu_3 = & \,\,\,\,\,\,\,\,\mathrm{m_2}\, {\mathrm{\dot \theta_1}}^2\, \ell_{2}^g\, \mathrm{\ell_1}\, \sin\!\left(\theta_2\right)
        \end{aligned}
        \phantom{\hspace{6cm}}
    \end{equation}
    
    Por fim, termos de referentes a atrito, gravitacional e acoplamento de entrada seguem em \eqref{eq:friction}, \eqref{eq:coupling} e \eqref{eq:input}.  
    
    \begin{equation}
        \mathbf{f}(\mathbf{\dot q}) = 
        \begin{bmatrix} 
        \mathrm{\mathrm{b_0}}\, \mathrm{\dot x}\\ \mathrm{\mathrm{b_1}}\, \mathrm{\dot \theta_1}\\ \mathrm{\mathrm{b_2}}\, \mathrm{\dot \theta_2} 
        \end{bmatrix}
        \label{eq:friction}
    \end{equation}
    
    \begin{equation}
        \mathbf{g}(\mathbf{q}) = 
        \begin{bmatrix} 
        0
        \\ \mathrm{m_2}\, g \,\left(\mathrm{\ell_1}\, \sin\!\left(\theta_1\right) + \ell_{2}^g\, \sin\!\left(\theta_1 + \theta_2\right)\right) + \mathrm{m_1}\, \ell_{1}^g\, \sin\!\left(\theta_1\right)\\ 
        \mathrm{m_2}\, g\, \ell_{2}^g\, \sin\!\left(\theta_1 + \theta_2\right) \end{bmatrix}
    \end{equation}
    
    \begin{equation}
        \mathbf{U}(\mathbf{q}) = \begin{bmatrix}
        1 \\
        0 \\
        0 
        \end{bmatrix}
        \label{eq:coupling}
    \end{equation}
    
    \begin{equation}
    \mathbf{u} = F
    \label{eq:input}
    \end{equation}
    
    A representação não-linear do sistema em espaço de estados assim como suas matrizes linearizadas não serão representadas integralmente neste documento por questão de brevidade. As matrizes linearizadas são abaixo.
    
    \begin{subequations}
    \begin{equation}
        \mathbf{A}(\mathbf{x_0}, \mathbf{u_0}) = 
        \begin{bmatrix}
        0 & 0 & 0 & 1 & 0 & 0 \\
        0 & 0 & 0 & 0 & 1 & 0 \\
        0 & 0 & 0 & 0 & 0 & 1 \\
        0 & A_{42} & A_{43} & 0 & 0 & 0 \\
        0 & A_{52} & A_{53} & 0 & 0 & 0 \\
        0 & A_{62} & A_{63} & 0 & 0 & 0 \\
        \end{bmatrix}
    \end{equation}
    \begin{equation}
        \mathbf{B}(\mathbf{x_0}, \mathbf{u_0}) = 
        \begin{bmatrix}
        0 \\
        0 \\
        0 \\
        B_4 \\
        B_5 \\
        B_6 \\
        \end{bmatrix}
    \end{equation}
        \begin{equation}
        \mathbf{C}(\mathbf{x_0}, \mathbf{u_0}) = 
        \begin{bmatrix}
        1 & 0 & 0 & 0 & 0 & 0\\
        0 & 0 & 1 & 0 & 0 & 0
        \end{bmatrix}
    \end{equation}
        \begin{equation}
        \mathbf{D}(\mathbf{x_0}, \mathbf{u_0}) = 
        \begin{bmatrix}
        0 \\
        0
        \end{bmatrix}
    \end{equation}
    \end{subequations}
    
    \clearpage
    \printbibliography

\end{document}
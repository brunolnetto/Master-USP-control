\section*{Exercício 7}
\addcontentsline{toc}{section}{Exercício 7}

    O controlador PID com possibilidades de controle anti-wind-up e pólo adicional para a parte derivativa apresenta esquematicamente a seguinte estrutura:
    
        \begin{equation}
            U(s) = P(s) + I(s) + D(s)
            \label{eq:PID}
        \end{equation}
    
    com 
    
        \begin{equation}
            P(s) = K_p E(s),
            \label{eq:P}
        \end{equation},
    
        \begin{equation}
            I(s, E(s), U(s), U_{sat}(s)) = \begin{cases}
            \frac{K_p}{T_i s} E(s) & \mbox{sem anti-windup} \\
            \frac{K_p}{T_i s} E(s) - \frac{K_p}{T_t s} \left(U(s) - U_{sat}(s)\right)& \mbox{com anti-windup}
            \end{cases}
            \label{eq:I}
        \end{equation},
    
        \begin{equation}
            D(s, E(s)) = 
            \begin{cases}
                K_p T_d s E(s), & \mbox{sem filtro} \\
                K_p T_d s \frac{1}{\frac{T_d}{N}s + 1} E(s), & \mbox{com filtro} \\
            \end{cases}
            \label{eq:D}
        \end{equation}
    
    Por meio da definição de $a, f \in \{0, 1\}$, para $a$ caso haja anti-windup e $f$ para pré-filtro, reescrevemos as equações \eqref{eq:I} e \eqref{eq:D} como
    
        \begin{equation}
            I(s, E(s)) =  \frac{K_p}{T_i s} E(s) - a \frac{K_p}{T_t s} \left( U(s) - U_{sat}(s)\right)
            \label{eq:Ia}
        \end{equation},
    
        \begin{equation}
            \begin{split}
                D(s, E(s)) &= K_p T_d s E(s) \left((1-f) + f \frac{1}{\frac{T_d}{N}s + 1}\right) \\
                & = K_p T_d s E(s) + f \left(\frac{s}{\frac{T_d}{N} s + 1}  - s\right) E(s) \\
                & = K_p T_d s E(s) + f \frac{\frac{T_d}{N} s^2}{\frac{T_d}{N}s + 1} E(s)
            \end{split}
            \label{eq:Df}
        \end{equation}
    
    Ao substituirmos as expressões \eqref{eq:P}, \eqref{eq:Df}, \eqref{eq:Df}, a expressão geral \eqref{eq:PID} consequentemente segue 
    
        \begin{equation}
            U(s) = K_p \left(1 + \frac{1}{T_i s} + T_d s + f \frac{\frac{T_d}{N} s^2}{\frac{T_d}{N}s+1}\right) E(s) +  a \frac{K_p}{T_t s} U_{sat}(s) - a \frac{K_p}{T_t s} U(s)
            \label{eq:PIDaf}
        \end{equation}
    
    Perceba que para $f=0$, o termo referente ao pré-filtro é nulo e para $a = 0$, o termo anti-windup é nulo. A fim de implementá-la, a expressão segue
    
        \begin{equation}
            \begin{split}
                U(s) &= K_p \left(\frac{T_t s}{T_t s + a} + \frac{T_t s}{T_t s + a} + \frac{T_d T_t s^2}{T_t s + a} + f \frac{\frac{T_d}{N} T_t s^3}{\left(\frac{T_d T_t}{N} s^2 + \left(\frac{T_d}{N} + T_t\right) s + a\right)} \right) E(s) + \frac{a K_p}{T_t s + a} U_{sat}(s) \\
                & \coloneqq \left(P'(s) + I'(s) + D'(s) + f F(s)\right) E(s) + a W(s) U_{sat}(s)
            \end{split}
            \label{eq:PIDfull}
        \end{equation}
    
    Como já explicitado na seção \emph{\nameref{sec:deducao7}}, convertemos cada uma das expressões $P'(s)$, $I'(s)$, $D'(s)$, $F(s)$ e $F(s)$ associadas aos termos dados pela sobreposição linear \eqref{eq:PIDfull}.
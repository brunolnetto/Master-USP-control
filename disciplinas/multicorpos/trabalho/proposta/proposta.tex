\documentclass[]{article}
\usepackage[brazilian]{babel}
\usepackage[utf8]{inputenc}
\usepackage{natbib}
\usepackage[backend=biber,style=authoryear]{biblatex}
\addbibresource{bibliography.bib}

%opening
\title{}
\author{Bruno Peixoto}

\begin{document}

\maketitle

Mecanismos de cadeia fechada são conhecidos pela sua destreza comparativamente a mecanismos de cadeia aberta. Aplicações comuns de utilização são usinagem, processo de \emph{pick and place} e entertenimento. O trabalho em questão tem por objetivo modelar  cinemática e dinâmcia de um mecanismo de cadeia fechada através dos métodos matemáticos apresentados na disciplina "Dinâmica de Sistemas Multicorpos e Suas Aplicações em
Robótica e Engenharia Veicular". Como referência, o aluno utilizará a síntese e modelagem do mecanismo apresentado em  \autocite{tubiblio52241}. Para fins práticos, o mecanismo foi construido e testado no laboratório de técnicas de automação da Universidade Técnica de Darmstadt. Sua construção está presente em \autocite{deskriptor2014}.

\printbibliography 

\end{document}

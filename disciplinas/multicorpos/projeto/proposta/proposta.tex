\documentclass[]{article}

\usepackage{fancyhdr}
\usepackage[brazilian]{babel}
\usepackage[utf8]{inputenc}
\usepackage{natbib}
\usepackage[backend=biber, style=alphabetic]{biblatex}

\addbibresource{bibliography.bib}

\newcommand{\studentname}{Bruno Peixoto}
\newcommand{\subjectname}{PME5236 Controle Digital de Sistemas Dinâmicos}
\newcommand{\uspid}{7206666}
\newcommand{\uspmail}{bruno.peixoto@usp.br}
\newcommand{\esnumber}{1}
\newcommand{\headerstyle}{\sffamily\bfseries\large}
\renewcommand{\headrulewidth}{1pt}

\setlength{\headheight}{14.0pt}
\fancyhead[LO, RE]{\headerstyle \exercisename \esnumber}
\fancyhead[RO, LE]{\headerstyle \subjectname}

\graphicspath{{./images/}}

\pagestyle{fancy}

%opening
\title{Proposta de projeto}
\author{\studentname \qquad \uspid \qquad \uspmail}

\begin{document}

\maketitle

Mecanismos de cadeia fechada são conhecidos pela sua destreza comparativamente a mecanismos de cadeia aberta. Aplicações comuns de utilização são usinagem, processo de \emph{pick and place} e entretenimento. O trabalho em questão tem por objetivo obter as equações cinemáticas e dinâmicas do mecanismo ilustrado em \autocite{deskriptor2014}. Como estado da arte, o aluno utilizará a síntese e modelagem do mecanismo apresentado em \autocite{tubiblio52241}. Para fins práticos, o mecanismo foi construido e testado no Instituto de Técnicas de Automação da Universidade Técnica de Darmstadt. Como sugestão da docência da disciplina, a modelgem proposta utiliza o método apresentado por \autocite{udwadia_kalaba_1996}, entretanto, por pesquisa complementar.

\printbibliography 

\end{document}

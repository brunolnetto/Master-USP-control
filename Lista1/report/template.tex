\documentclass{article}
\usepackage{graphicx,fancyhdr,amsmath,amssymb,amsthm,subfig,url,hyperref}
\usepackage[margin=1in]{geometry}
\usepackage[brazilian]{babel}
\usepackage[utf8]{inputenc}
\usepackage{mathtools}

%----------------------- Macros and Definitions --------------------------

%%% FILL THIS OUT
\newcommand{\studentname}{Bruno H. L. N. Peixoto}
\newcommand{\uspid}{7206666}
\newcommand{\uspmail}{bruno.peixoto@usp.br}
\newcommand{\esnumber}{1}
%%% END

\renewcommand{\theenumi}{\bf \Alph{enumi}}

\fancypagestyle{plain}{}
\pagestyle{fancy}
\fancyhf{}
\fancyhead[RO,LE]{\sffamily\bfseries\large Universidade de São Paulo}
\fancyhead[LO,RE]{\sffamily\bfseries\large PTC5611 Controle Digital de Sistemas Dinâmicos}
\renewcommand{\headrulewidth}{1pt}

\graphicspath{{figures/}}

%-------------------------------- Title ----------------------------------

\title{Lista de exercícios \esnumber}
\author{\studentname \qquad Número USP: \uspid \qquad E-mail USP: \uspmail}

%--------------------------------- Text ----------------------------------

\begin{document}
\maketitle

\section*{Exercício 1}
\begin{enumerate}
\item %A

Por hipótese, a frequência de amostragem utilizada satisfaz o critério de Nyquist \footnote{$\omega_s \geq 2\omega_0$}. Como $t = kT_s = \frac{k}{f_s}$:

\begin{equation}
\label{eq:ex1a}
x[k] = cos(k \frac{\omega}{f_s})  \stackrel{!}{=} cos(k \alpha) \Longleftrightarrow \alpha = \frac{\omega}{f_s}
\end{equation}

\begin{equation}
\therefore \omega = 250\pi \frac{rad}{s}
\end{equation}

\item %B
Por (\ref{eq:ex1a}), conclui-se que $f_s = 12$ $kHz$. Dado que a frequência do sistema é $4000\pi \frac{rad}{s}$, pelo teorema de Nyquist $\omega_s > 8000\pi \frac{rad}{s} \approx 24000 \frac{rad}{s}$. Devido à violação, não é possível reconstruir o sinal por meio de um passa-baixas ideal. 

\item %C
42

\end{enumerate}

\section*{Exercício 2}
\begin{enumerate}
\item %A

\begin{itemize}
	\item $Z\{\sum_{i=0}^{n}x[i] \} = \frac{1}{1-z^{-1}}X(z)$
	
	\begin{equation}
	\begin{split}
	Z\{\sum_{i=0}^{n}x[i] \} & = \sum_{n=0}^{\infty} z^{-n} \sum_{i=0}^{n} x[i] \\
	& = x[0] + z^{-1} \sum_{i=0}^{1} x[i] + z^{-2} \sum_{i=0}^{2} x[i] + \cdots \\
	& = x[0] \sum_{i=0}^{\infty} z^{-i} + x[1] \sum_{i=1}^{\infty} z^{-i} + \cdots \\
	\end{split}
	\end{equation}

	Como

	\begin{equation}
	\label{eq:pgsoma}
	\sum_{i=k}^{\infty} z^{-i} \coloneq \frac{z^{-k}}{1 - z^{-1}}, |z|<1
	\end{equation}
	
	Então
	
	\begin{equation}
	\begin{split}
	x[0] \sum_{i=0}^{\infty} z^{-i} + x[1] \sum_{i=1}^{\infty} z^{-i} + \cdots & = 
	x[0] \frac{1}{1 - z^{-1}} + x[1] \frac{z^{-1}}{1 - z^{-1}} + \cdots \\
	& = \frac{1}{1 - z^{-1}} \underbrace{\sum_{i=0}^{\infty} x[i].z^{-i}}_{\coloneqq X(z)}
	\end{split}
	\end{equation}
	
	\begin{equation}
	Z\{\sum_{i=0}^{n}x[i] \} = \frac{1}{1-z^{-1}}.X(z) \hspace{10pt} \blacksquare
	\end{equation}
	
	\item $Z\{\sum_{i=0}^{n}x[i-1] \} = \frac{z^{-1}}{1-z^{-1}}X(z)$

	\begin{equation}
	\begin{split}
	Z\{\sum_{i=0}^{n}x[i] \} & = \sum_{n=0}^{\infty} z^{-n} \sum_{i=0}^{n} x[i-1] \\
	& = x[-1] + z^{-1} \sum_{i=0}^{1} x[i-1] + z^{-2} \sum_{i=0}^{2} x[i-1] + \cdots \\
	& = \underbrace{x[-1]}_{\coloneqq 0} \sum_{i=0}^{\infty} z^{-i} + x[0] \sum_{i=1}^{\infty} z^{-i} + \cdots \\
	& 
	\end{split}
	\end{equation}

	\item $\lim\limits_{z \rightarrow 1} X(z) = \sum_{i=0}^{\infty}x[i]$
	
\end{itemize}


\item %B

\item %C

\item %D

\item %E

\end{enumerate}


\end{document}
